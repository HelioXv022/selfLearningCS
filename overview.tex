\documentclass{article}

% Language setting
% Replace `english' with e.g. `spanish' to change the document language
\usepackage[english]{babel}

% Set page size and margins
% Replace `letterpaper' with `a4paper' for UK/EU standard size
\usepackage[letterpaper,top=2cm,bottom=2cm,left=3cm,right=3cm,marginparwidth=1.75cm]{geometry}

% Useful packages
\usepackage{amsmath}
\usepackage{graphicx}
\usepackage[colorlinks=true, allcolors=blue]{hyperref}

\title{Computer Science: An Overview}
\author{Ziqi Xv}

\begin{document}
\maketitle

\section{Intro}
I wonder what will happen if I put my smart phone in a rotating system while using its navigating function. Will the GPS system crash? Use the accelerator to do simulation of this scenario.

\section{Flip-flop}
If I want to store a string of zeros and ones in a register, eg. 00000101, which is 5 represneted by a byte, then I can input 8 pulses into 8 flip-flops as down-down-down-down-down-up-down-up. Well, the question is, how to build a flip-flop through physical implements? Notice that a register is made of many flip-flops.\\
What if the two inputs are simultaneously set 1?

\end{document}
